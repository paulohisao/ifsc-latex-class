% Classe para teses e dissertações no IFSC - USP
%
% 
% autor: Thiago S. Mosqueiro
% e-mail contato: thiago -dot- mosqueiro (at) ursa.ifsc.usp.br
% Site com informações sobre a classe:
% http://thmosqueiro.vandroiy.com/ifsc-latex/
%
% Se você estiver chamando latex + dvi2ps + ps2pdf, sua folha sairá em letter (não a4)
% a menos que você: (i) adicione -sPAPERSIZE=a4 à chamada do ps2pdf ou (ii) configure
% como saída padrão o papel a4.
%
%
% Bom uso de LaTeX a todos!!
%


\documentclass[a4paper, espaco=emeio, dvips, plainheader, twoside, openright, final, normalfigtabnum, tocpage=plain, titulodireita]{ifsc}
  
  
  \usepackage{additionals}

% A opção linktocpage é muito importante para poder fazer com que TOC tenha
% quebra de linhas!!!
  \usepackage[colorlinks, hyperindex, breaklinks]{hyperref}
  \hypersetup
  {
    pdftitle = {T\'{\i}tulo da sua tese},
    pdfauthor = {Nome do autor},
    pdfsubject = {Assunto da tese},
    pdfkeywords = {Palavras-chave},
    pdfcreator = {Como você compilou o tex?},
    linkbordercolor = {1 0 0},
    pdftoolbar = true,
    pdfmenubar = true,
    colorlinks = false,
    citecolor=black,
    linkcolor=black,
    urlcolor=black,
  }

  \makeindex
  
  \def\@cite#1#2{({#1\if@tempswa:#2\fi})}
  
  \setlength{\footnotemargin}{0in}


  % Pasta onde voce vai deixar suas imagens
  \graphicspath{{figuras/}}

 % Caso seja a versão corrigida, descomente abaixo.
 % \renewcommand{\IFSCversao}{Vers\~ao Corrigida\\ (versão original disponível na Unidade que aloja o Programa)}

  \usepackage{tikz}
  
  % Neste arquivo você pode inserir suas definições (veja exemplo)
  \include{arquivos_tex/definicoes}

  \begin{document}


    \include{arquivos_tex/titlepage}

    \include{arquivos_tex/aknowledgements}

    \include{arquivos_tex/resumo}

    \listoffigures
    \newpage

    \listoftables
    \newpage

    \tableofcontents
    \newpage

    \include{arquivos_tex/intro}
	
    \bibliographystyle{ifsc_abnt}
    \bibliography{teste.bib}

    \appendix
    
    \chapter{Um ap\^endice qualquer}

Lorem ipsum dolor sit amet, consectetur adipiscing elit. Nunc vitae accumsan nisl. In aliquet, risus nec semper lacinia, neque augue gravida urna, non consectetur eros massa a ipsum. Praesent iaculis auctor nunc, non lobortis eros malesuada vitae. Vivamus arcu arcu, pulvinar eu rutrum in, gravida vel metus. Donec libero tellus, porttitor in consequat eget, varius quis turpis. Mauris hendrerit sagittis tellus, rutrum rutrum enim euismod vitae. Aliquam sit amet lorem metus. Maecenas neque elit, fringilla quis fringilla at, vulputate a felis. Phasellus facilisis magna dictum quam sodales vel imperdiet turpis sodales. Integer non metus ut arcu volutpat condimentum. Nunc turpis mi, adipiscing et fringilla ut, eleifend nec libero. Aenean molestie accumsan sapien, et egestas sapien varius ac.

\section{Primeira se\c{c}\~ao da sua tese}

Mauris eu tellus metus. Donec placerat sagittis lorem, et rhoncus diam semper tempus. Aliquam mattis tellus et nulla dignissim non mattis augue fermentum. Nullam lobortis bibendum nulla ut condimentum. Sed sed turpis at magna lacinia semper id ac risus. Praesent bibendum risus ac nulla feugiat molestie. Morbi non gravida sem. In a tristique velit. In congue dolor enim, quis blandit nulla. Nulla bibendum tortor neque.

Proin vitae dolor non ipsum iaculis tempor. Nunc eget sapien nisi. Nullam lobortis urna purus. Curabitur sed eros dui, sed elementum felis. Donec vel erat ligula. Pellentesque in libero et nibh congue dapibus. In tristique nunc nec enim suscipit ultricies.

Sed nec aliquam mauris. Curabitur placerat, risus ac gravida bibendum, metus tortor aliquet eros, in vestibulum ipsum erat eu nibh. Aenean sit amet purus magna. Fusce euismod, ligula ac consequat tempus, sapien turpis tempor est, quis pulvinar enim risus eget massa. Nunc ornare, purus in pulvinar tempor, nunc dui hendrerit ipsum, condimentum consequat quam lorem id nisi. Cras feugiat neque sed eros vulputate fermentum rutrum felis aliquet. Nam mollis commodo volutpat. In a interdum mi. Aliquam a nisl risus. In sem turpis, bibendum et venenatis a, venenatis et urna.

Duis eleifend viverra neque, id auctor libero fermentum vitae. Proin eu nibh turpis, ac fringilla neque. Nam mattis commodo rhoncus. Donec elementum malesuada ultrices. Sed viverra, velit at elementum porttitor, erat nisl ultricies diam, ac eleifend ipsum lacus in sapien. Duis sit amet mi felis. Nunc dignissim nibh mauris. Nunc porttitor fringilla tristique. Cras vel risus in tellus tristique tempus. Nam eu eros erat. Praesent sed metus nisl, ac tempor lorem. Aenean at velit vitae ante feugiat laoreet ac eu risus.

Nulla facilisi. Suspendisse id massa eu nisi malesuada luctus. Morbi feugiat eros id magna tristique accumsan. Phasellus elementum nunc id est imperdiet ac congue justo molestie. Morbi lobortis gravida velit, nec varius turpis vestibulum nec. Donec nec nisi sit amet risus lobortis bibendum. Aliquam turpis sem, rutrum nec mattis non, facilisis sed tellus. Aenean ut risus neque. Sed mattis, tortor non ultrices convallis, purus erat condimentum libero, id tristique elit metus vitae tellus. Sed laoreet mauris nec lacus pretium facilisis. Donec sem ipsum, porta quis laoreet quis, adipiscing ac tellus. Duis vestibulum, lorem ac rutrum varius, urna orci pretium sem, non luctus dolor ipsum et tortor. Aenean et fringilla lectus. In vitae velit libero, malesuada posuere augue. Fusce nisi sem, lobortis non malesuada at, consectetur ac sapien. Cras eleifend fringilla neque, non interdum sapien interdum rhoncus.
  
  \end{document}
